\documentclass{noteformyself}

\newcommand{\Yang}[1]{\textcolor{red}{Yang: #1}} % 批注命令
../../template/notation.tex            % 加载符号定义文件


\title{Basepoint Free Theorem for Nef and Big Line Bundle on Positive Characteristic}
\author{Tianle Yang}
\date{\today}
\authorinfo{\href{www.tianleyang.com}{My Homepage}}

% % available on the mode "\sectionlevel=chapter" or "\sectionlevel=book".
% \setCJKfamilyfont{lxgwwenkai}{LXGW WenKai} % 定义霞鹜文楷,若未安装,请去掉相关代码编译或使用其他字体
% \coversentence{\CJKfamily{lxgwwenkai}阿巴阿巴阿巴阿巴阿巴阿巴阿巴阿巴!}
% \coverimage{placeholder.jpg} % 封面图片,
% % \coverlinecolor{red} % 封面线条颜色,available on the mode "\sectionlevel=chapter" or "\sectionlevel=book".
% \covertitlefont{Allura} % 封面标题字体,available on the mode "\sectionlevel=chapter" or "\sectionlevel=book".
% % \covertextcolor{green} % 封面文字颜色,available on the mode "\sectionlevel=chapter" or "\sectionlevel=book".


\begin{document}

    \maketitle

    % \tableofcontents % 生成目录

    \section{Definition and Examples}

    In this section and all the books, we denote by \(\calh\) the upper half plane, and let \(\SL_2(\bbz)\) act on \(\calh\) in the natural way.

    \begin{definition}\label{def: modular form}
        Let \(\Gamma < \SL_2(\bbz)\) be a subgroup of finite index. 
        A \emph{modular form} of \emph{weight} \(k\) and \emph{level} \(\Gamma\) is a holomorphic function \(f: \calh \to \bbc\) such that:
        \begin{enumerate}
            \item (Automorphy) For all \(\gamma = \mat{a}{b}{c}{d} \in \Gamma\), we have \(f(\gamma(z)) = (cz + d)^k f(z) \).
            \item (Holomorphic at the cusps) For every \(\gamma \in \Gamma\), the function \((cz + d)^{-k} f(\gamma(z))\) is bounded as \(\Im(z) \to +\infty\).
        \end{enumerate}
    \end{definition} % 这里是正文

    \printbibliography[heading=bibintoc, title={References}] % 打印参考文献
\end{document}